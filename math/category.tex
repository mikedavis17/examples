\documentclass[12pt]{article}
\usepackage{times}
\usepackage{graphicx}
\usepackage{float}
\usepackage{amsfonts}

% Examples go in boxes
\floatstyle{boxed}
\newfloat{example}{h}{loe}
\floatname{example}{Example}

% BEGIN FORMATTING
\baselineskip=18pt
\hsize=6in
%\topmargin=-0.25in
\textheight=8.5in
\evensidemargin=0.0in
\oddsidemargin=0.0in
\pagestyle{myheadings}
\markright{Some notes and exercises in category theory.}

\bibliographystyle{plain}

% BEGIN MACRO DEFINITIONS

\begin{document}

\title{Category Theory Notes}
\author{Mike Davis}
%%\date{February 1, 2007}
\maketitle

%%\tableofcontents\newpage
%%\listoffigures\newpage
%%\listoftables\newpage

\section{Session 10}
\subsection{Exercise 4}
\subsubsection{(a)}
This exercise is a special case of part (c) below.  Let $n = 1.$  The
endpoints Buffalo and Rochester, comprise $S^1$, The interval $I$ is
$B^1$, and the interior of the line segment is $O^1$.  The road $R$ is
$\mathbb{R}^1$.  The inclusion maps and point maps are as defined.  Function
$m$ corresponds to function $x$ below.  Function $y$ is as described
below.  The conclusion says that at some point my car and your car
will be at the same point in $R$.

\subsubsection{(c)}

Let $n$ be any positive integer.
Let $S^n$ be the $n$-dimensional sphere.  
Let $B^n$ be the $n$-dimensional ball.  
Define $O^n$ to be the interior of the $n$-dimensional ball,
i.e., $B^n \setminus S^n$.  
Let $\emph{1}$ be the terminal object.
$\mathbb{R}^n$ is the $n$-dimensional Euclidean space.

Define inclusion maps 
$j_n : S^n \rightarrow B^n$,
$i_n : B^n \rightarrow \mathbb{R}^n$,
and 
$k_n : O^n \rightarrow B^n$
from their respective domains to codomains.

Define families of maps
$\{P\} : \emph{1} \rightarrow S^n$
and 
$\{Q\} : \emph{1} \rightarrow O^n$
defining points in their respective codomains.

Let $x$ be any continuous map 
$x : B^n \rightarrow \mathbb{R}^n$ 
such that for any 
$p \in \{P\}$,
\[x \circ j_n \circ p = i_n \circ j_n \circ p.\]
In other words, every point on the boundary maps to itself.

Let $y$ be any continuous map 
$y : B^n \rightarrow \mathbb{R}^n$ 
such that for any
$p \in \{P\}$ 
there exists 
$q \in \{Q\}$,
such that 
\[y \circ j_n \circ p = i_n \circ k_n \circ q.\]
In other words, every point on the boundary maps to a point in the
interior.

Then there exists $q \in \{Q\}$ such that 
\[x \circ k_n \circ q = y \circ k_n \circ q,\]
which is to say that there is a point in the interior of the ball
which both $x$ and $y$ map to the same point in $\mathbb{R}^n$.

\end{document}
